\documentclass{article}
\usepackage{titlesec}

%Paquetes
\usepackage[left=4cm, right=4cm]{geometry}
\usepackage{afterpage}
\usepackage{palatino}%Fuente
 \usepackage{eulervm}%Fuente
\usepackage{graphicx}%Imágenes
\usepackage{float}%Imágenes
\usepackage{subcaption}%Imágenes
\usepackage{enumitem}%Listas
\usepackage{parskip}%Espacio entre párrafos
\usepackage{multicol}
\usepackage{amsthm,thmtools,xcolor}
\usepackage{amssymb}%Mate
\usepackage{amsmath}%Mate
\usepackage{tikz}%Mate (diagramas)
\usepackage{dutchcal}
\usepackage{tikz-cd}
\usepackage{xcolor}
\definecolor{blue-violet}{rgb}{0.54, 0.17, 0.89}
\usetikzlibrary{%
	matrix,%
	calc,%
	arrows,%
	shapes,
	decorations.markings,backgrounds,calc,intersections
}
\usepackage[bookmarks,bookmarksopen,bookmarksdepth=3]{hyperref}%Links a lugares en el texto
\hypersetup{%colores
	colorlinks=true,
	urlcolor=blue,
	linkcolor=magenta,
	citecolor=blue,
	filecolor=blue,
	urlbordercolor=white,
	linkbordercolor=white,
	citebordercolor=white,
	filebordercolor=white
}
\usepackage{cleveref}
\Crefname{exercise}{Exercise}{Exercises}

\newcommand{\fakesection}[1]{%
	\par\refstepcounter{section}% Increase section counter
	\sectionmark{#1}% Add section mark (header)
	\addcontentsline{toc}{section}{\protect\numberline{\thesection}#1}% Add section to ToC
	% Add more content here, if needed.
}

\makeatletter %Hide section number
\def\@seccntformat#1{%
	\expandafter\ifx\csname c@#1\endcsname\c@section\else
	\csname the#1\endcsname\quad
	\fi}
\makeatother
\usepackage{sectsty}
\sectionfont{\fontsize{17}{20}\selectfont}


%Referencias
%\usepackage[style=authortitle,backend=bibtex]{biblatex}
%\addbibresource{exercises.bib}

\definecolor{blue-violet}{rgb}{0.54, 0.17, 0.89}
\definecolor{azure}{rgb}{0.0, 0.5, 1.0}
\definecolor{green(ncs)}{rgb}{0.0, 0.62, 0.42}
\definecolor{forestgreen}{rgb}{0.13, 0.55, 0.13}
\definecolor{limegreen}{rgb}{0.2, 0.8, 0.2}
\definecolor{palatinateblue}{rgb}{0.15, 0.23, 0.89}
\definecolor{trueblue}{rgb}{0.0, 0.45, 0.81}
\definecolor{goldenyellow}{rgb}{1.0, 0.87, 0.0}
\definecolor{fashionfuchsia}{rgb}{0.96, 0.0, 0.63}
\definecolor{brightcerulean}{rgb}{0.11, 0.67, 0.84}
\definecolor{jonquil}{rgb}{0.98, 0.85, 0.37}
\definecolor{lavendermagenta}{rgb}{0.93, 0.51, 0.93}
\definecolor{peru}{rgb}{0.8, 0.52, 0.25}
\definecolor{persimmon}{rgb}{0.93, 0.35, 0.0}
\definecolor{persianred}{rgb}{0.8, 0.2, 0.2}
\definecolor{persianblue}{rgb}{0.11, 0.22, 0.73}
\definecolor{persiangreen}{rgb}{0.0, 0.65, 0.58}
\definecolor{persianyellow}{rgb}{0.9, 0.89, 0.0}

%\theoremstyle{definition}

\declaretheoremstyle[headfont=\color{trueblue}\normalfont\bfseries,]{colored1}
\declaretheoremstyle[headfont=\color{forestgreen}\normalfont\bfseries,]{colored2}
\declaretheoremstyle[headfont=\color{peru}\normalfont\bfseries,]{colored3}
\declaretheoremstyle[headfont=\color{persiangreen}\normalfont\bfseries,]{colored4}
\declaretheoremstyle[headfont=\color{brightcerulean}\normalfont\bfseries,]{colored5}
\declaretheoremstyle[headfont=\color{lavendermagenta}\normalfont\bfseries,]{colored6}
\declaretheoremstyle[headfont=\color{blue-violet}\normalfont\bfseries,]{colored7}
\declaretheoremstyle[headfont=\color{green(ncs)}\normalfont\bfseries,]{colored8}
\declaretheoremstyle[headfont=\color{peru}\normalfont\bfseries,]{colored9}
\declaretheoremstyle[headfont=\color{persiangreen}\normalfont\bfseries,]{colored10}

\declaretheorem[style=colored1,numberwithin=section,name=Theorem]{thm}
\declaretheorem[style=colored2,numberwithin=section,numberlike=thm,name=Proposition]{prop}
\declaretheorem[style=colored3,numberwithin=section,numberlike=thm,name=Lemma]{lemma}
\declaretheorem[style=colored4,numberwithin=section,numberlike=thm,name=Corollary]{coro}
\declaretheorem[style=colored5,numbered=no,name=Example]{example}
\declaretheorem[style=colored5,numbered=no,name=Examples]{exemplos}
\declaretheorem[style=colored7,numberwithin=section,name=Exercise]{exercise}
\declaretheorem[style=colored9,numberwithin=section,name=Remark]{remark}
\declaretheorem[style=colored9,numbered=no,name=Claim]{claim}
\declaretheorem[style=colored8,numbered=no,name=Definition]{defn}
\declaretheorem[style=colored10,numbered=no,name=Question]{question}

\numberwithin{equation}{section}

\newcommand{\R}{\mathbb{R}}
\newcommand{\Z}{\mathbb{Z}}
\newcommand{\N}{\mathbb{N}}
\newcommand{\C}{\mathbb{C}}
\newcommand{\Q}{\mathbb{Q}}
\newcommand{\D}{\mathbb{D}}
\newcommand{\T}{\mathbb{T}}
\renewcommand{\P}{\mathbb{P}}
\newcommand{\Ac}{\mathcal{A}}
\newcommand{\Bc}{\mathcal{B}}
\newcommand{\Cc}{\mathcal{C}}
\newcommand{\Dc}{\mathcal{D}}
\newcommand{\Ec}{\mathcal{E}}
\newcommand{\Fc}{\mathcal{F}}
\newcommand{\Gc}{\mathcal{G}}
\newcommand{\Lc}{\mathcal{L}}
\newcommand{\Oc}{\mathcal{O}}
\newcommand{\Qc}{\mathcal{Q}}
\newcommand{\Sc}{\mathcal{S}}
\newcommand{\Wc}{\mathcal{W}}
\newcommand{\mf}{\mathfrak{m}}
\newcommand{\gf}{\mathfrak{g}}
\newcommand{\X}{\mathfrak{X}}
\newcommand{\hf}{\mathfrak{h}}
\newcommand{\glf}{\mathfrak{gl}}
\newcommand{\of}{\mathfrak{o}}

\renewcommand{\Im}{\operatorname{Im}}
\renewcommand{\O}{\operatorname{O}}
\renewcommand{\S}{\mathbb{S}}
\renewcommand{\T}{\mathbb{T}}
\DeclareMathOperator{\Lie}{\operatorname{Lie}}

\DeclareMathOperator{\img}{img}
\DeclareMathOperator{\Arg}{Arg}
\DeclareMathOperator{\End}{End}
\DeclareMathOperator{\I}{I}
\DeclareMathOperator{\id}{id}
\DeclareMathOperator{\Alt}{Alt}
\DeclareMathOperator{\sgn}{sgn}
\DeclareMathOperator{\supp}{supp}
\DeclareMathOperator{\Int}{Int}
\DeclareMathOperator{\Ob}{Ob}
\DeclareMathOperator{\Mor}{Mor}
\DeclareMathOperator{\Top}{Top}
\DeclareMathOperator{\CGWH}{CGWH}
\DeclareMathOperator{\Hom}{Hom}
\DeclareMathOperator{\Map}{Map}
\DeclareMathOperator{\Tot}{Tot}
\DeclareMathOperator{\Vect}{Vect}
\DeclareMathOperator{\VectBund}{VectBund}
\DeclareMathOperator{\Open}{Open}
\DeclareMathOperator{\Ring}{Ring}
\DeclareMathOperator{\Set}{Set}
\DeclareMathOperator{\GL}{GL}
\DeclareMathOperator{\SL}{SL}
\DeclareMathOperator{\SO}{SO}
\DeclareMathOperator{\U}{U}
\DeclareMathOperator{\SU}{SU}
\DeclareMathOperator{\Sp}{Sp}
\DeclareMathOperator{\M}{M}
\DeclareMathOperator{\Aut}{Aut}
\DeclareMathOperator{\PGL}{PGL}
\DeclareMathOperator{\PSL}{PSL}
\DeclareMathOperator{\St}{St}

\begin{document}
\begin{minipage}{\textwidth}
	\begin{minipage}{.5\textwidth}
		Complex Manifolds in Dimension 1
	\end{minipage}%
	\begin{minipage}{.5\textwidth}
		\raggedleft
		Daniel González Casanova Azuela\par
		{\small\href{https://github.com/danimalabares/riemann-surfaces}{github.com/danimalabares/riemann-surfaces}}
	\end{minipage}%
\end{minipage}\vspace{.2cm}\hrule
\section{Home Assignment 3: Lie groups}
\setcounter{section}{3}
\begin{defn}
	A \textit{\textbf{Lie group}} is a smooth manifold equipped with a group structure such that the group operations are smooth. Lie group $G$ \textbf{\textit{acts on a manifold}} $M$ if the group action is given by the smooth map $G \times M \to M$.
\end{defn}
\begin{exercise}
	Prove that $\SL(n,\R)$ is a Lie group. Prove that it is connected.
\end{exercise}
\begin{proof}
	Recall that $\SL(n,\R)$ is the subgroup of $\GL(n,\R)$ of matrices with determinant $1$, so it is the preimage of $\{1\}$ under the smooth function $\det:\GL(n,\R)\to\R$. In fact, $1$ is a regular value of $\det$ because $\det$ is surjective and of constant rank $\equiv1$, making $\SL(n,\R)$ a submanifold. (Of course, $\GL(n,\R)$ is a submanifold of $\R^{2n}=\M(n,\R)$ because it is an open subset, namely, the preimage of $\R\backslash0$ under the continuous function $\det$.)
	
	Moreover, we may think of $\det$ as a group homomorphism from $\GL(n,\R)$ to the multiplicative group $\R\backslash0$, so that $\SL(n,\R)=\ker\det$, making it a subgroup. The restriction of the group operations from $\GL(n,\R)$ are smooth, making $\SL(n,\R)$ a Lie group.
\end{proof}

\begin{exercise}
	Prove that the special unitary group $\SU(n)$ acts transitively on the projective space $\C P^{n-1}$. Find the stabilizer $\St_x(\SU(n))$ of a point $x \in \C P^{n-1}$. Prove that it is connected, or find a counterexample.
\end{exercise}
\begin{proof}
	%Choose any two points $\C P^{n-1}$, say, $x=z_1:\ldots:z_n$ and $y=w_1:\ldots:w_n$. Any  matrix $U\in\SU(n)$ is 
	Any point in $\C P^{n-1}$ has two representants in the set of points of $\C^n$ of norm 1. Indeed, suppose $x=z_1:\ldots:z_n$ is a point of $\C P^{n-1}$. Since not all coordinates are zero, we may normalize dividing by $\sqrt{z^2_1+\ldots+z^2_n}$. But of course the point $(-z_1,\ldots,-z_n)\in\C^n$ is also a representant of $x$ that has norm 1.
	
	Anyway, a matrix $U\in\SU(n)$ will preserve
\end{proof}

\begin{defn}
	Let $W$ be an $n$-dimensional complex vector space equipped with a complex-linear non-degenerate quadratic form $s$. Consider the \textbf{\textit{complex orthogonal group}} $\O(n,\C)$ of all matrices $A \in \GL(W )$ preserving $s$. A subspace $V \subset W$ is called \textbf{\textit{isotropic}} if $s|_W = 0$. It is called \textbf{\textit{maximally isotropic}}, or \textbf{\textit{Lagrangian}}, if $\dim V = [n/2]$.
\end{defn}

\begin{exercise}
	Prove that $\SO(n, \C) := \O(n, \C) \cap \SL(n, \C)$ is a Lie group
	which has index 2 in $\O(n, \C)$. Prove that it is connected.
\end{exercise}
\begin{proof}
	content...
\end{proof}
\end{document}