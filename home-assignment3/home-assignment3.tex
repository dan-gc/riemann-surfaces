\documentclass{article}
\usepackage{titlesec}

\usepackage[citestyle=authortitle-terse,backend=bibtex]{biblatex}
\addbibresource{bibliography.bib}

\setcounter{secnumdepth}{0}
\usepackage{sectsty}
\sectionfont{\fontsize{17}{20}\selectfont}

\usepackage[left=4cm, right=4cm]{geometry}
\usepackage{palatino,eulervm,dutchcal,xcolor}%fonts
\usepackage{graphicx,subcaption,float}
\usepackage{enumitem,parskip,multicol}
\usepackage{amsthm,amssymb,amsmath,mathtools,thmtools}
\usepackage{tikz,tikz-cd}
\usetikzlibrary{%
	matrix,%
	calc,%
	arrows,%
	shapes,
	decorations.markings,backgrounds,calc,intersections}
\tikzcdset{scale cd/.style={every label/.append style={scale=#1},
		cells={nodes={scale=#1}}}}
\usepackage[bookmarks,bookmarksopen,bookmarksdepth=3]{hyperref}
\hypersetup{%colores
	colorlinks=true,
	urlcolor=blue,
	linkcolor=magenta,
	citecolor=blue,
	filecolor=blue,
	urlbordercolor=white,
	linkbordercolor=white,
	citebordercolor=white,
	filebordercolor=white}
\usepackage{cleveref}
\Crefname{exercise}{Exercise}{Exercises}

\newcommand{\fakesection}[1]{%
	\par\refstepcounter{section}% Increase section counter
	\sectionmark{#1}% Add section mark (header)
	\addcontentsline{toc}{section}{\protect\numberline{\thesection}#1}% Add section to ToC
	% Add more content here, if needed.
}

\makeatletter %Hide section number
\def\@seccntformat#1{%
	\expandafter\ifx\csname c@#1\endcsname\c@section\else
	\csname the#1\endcsname\quad
	\fi}
\makeatother

\definecolor{blue-violet}{rgb}{0.54, 0.17, 0.89}
\definecolor{azure}{rgb}{0.0, 0.5, 1.0}
\definecolor{green(ncs)}{rgb}{0.0, 0.62, 0.42}
\definecolor{forestgreen}{rgb}{0.13, 0.55, 0.13}
\definecolor{limegreen}{rgb}{0.2, 0.8, 0.2}
\definecolor{palatinateblue}{rgb}{0.15, 0.23, 0.89}
\definecolor{trueblue}{rgb}{0.0, 0.45, 0.81}
\definecolor{goldenyellow}{rgb}{1.0, 0.87, 0.0}
\definecolor{fashionfuchsia}{rgb}{0.96, 0.0, 0.63}
\definecolor{brightcerulean}{rgb}{0.11, 0.67, 0.84}
\definecolor{jonquil}{rgb}{0.98, 0.85, 0.37}
\definecolor{lavendermagenta}{rgb}{0.93, 0.51, 0.93}
\definecolor{peru}{rgb}{0.8, 0.52, 0.25}
\definecolor{persimmon}{rgb}{0.93, 0.35, 0.0}
\definecolor{persianred}{rgb}{0.8, 0.2, 0.2}
\definecolor{persianblue}{rgb}{0.11, 0.22, 0.73}
\definecolor{persiangreen}{rgb}{0.0, 0.65, 0.58}
\definecolor{persianyellow}{rgb}{0.9, 0.89, 0.0}

\declaretheoremstyle[headfont=\color{trueblue}\normalfont\bfseries,]{colored1}
\declaretheoremstyle[headfont=\color{forestgreen}\normalfont\bfseries,]{colored2}
\declaretheoremstyle[headfont=\color{peru}\normalfont\bfseries,]{colored3}
\declaretheoremstyle[headfont=\color{persiangreen}\normalfont\bfseries,]{colored4}
\declaretheoremstyle[headfont=\color{brightcerulean}\normalfont\bfseries,]{colored5}
\declaretheoremstyle[headfont=\color{lavendermagenta}\normalfont\bfseries,]{colored6}
\declaretheoremstyle[headfont=\color{blue-violet}\normalfont\bfseries,]{colored7}
\declaretheoremstyle[headfont=\color{green(ncs)}\normalfont\bfseries,]{colored8}
\declaretheoremstyle[headfont=\color{peru}\normalfont\bfseries,]{colored9}
\declaretheoremstyle[headfont=\color{persiangreen}\normalfont\bfseries,]{colored10}

\declaretheorem[style=colored1,numberwithin=section,name=Theorem]{thm}
\declaretheorem[style=colored2,numberwithin=section,numberlike=thm,name=Proposition]{prop}
\declaretheorem[style=colored3,numberwithin=section,numberlike=thm,name=Lemma]{lemma}
\declaretheorem[style=colored4,numberwithin=section,numberlike=thm,name=Corollary]{coro}
\declaretheorem[style=colored5,numbered=no,name=Example]{example}
\declaretheorem[style=colored5,numbered=no,name=Examples]{exemplos}
\declaretheorem[style=colored7,numberwithin=section,name=Exercise]{exercise}
\declaretheorem[style=colored9,numbered=no,name=Remark]{remark}
\declaretheorem[style=colored9,numbered=no,name=Claim]{claim}
\declaretheorem[style=colored8,numbered=no,name=Definition]{defn}
\declaretheorem[style=colored10,numbered=no,name=Question]{question}

\newcommand{\R}{\mathbb{R}}
\newcommand{\Z}{\mathbb{Z}}
\newcommand{\N}{\mathbb{N}}
\newcommand{\C}{\mathbb{C}}
\newcommand{\Q}{\mathbb{Q}}
\newcommand{\D}{\mathbb{D}}
\newcommand{\T}{\mathbb{T}}
\renewcommand{\P}{\mathbb{P}}
\newcommand{\Ac}{\mathcal{A}}
\newcommand{\Bc}{\mathcal{B}}
\newcommand{\Cc}{\mathcal{C}}
\newcommand{\Dc}{\mathcal{D}}
\newcommand{\Ec}{\mathcal{E}}
\newcommand{\Fc}{\mathcal{F}}
\newcommand{\Gc}{\mathcal{G}}
\newcommand{\Lc}{\mathcal{L}}
\newcommand{\Oc}{\mathcal{O}}
\newcommand{\Qc}{\mathcal{Q}}
\newcommand{\Sc}{\mathcal{S}}
\newcommand{\Wc}{\mathcal{W}}
\newcommand{\mf}{\mathfrak{m}}
\newcommand{\gf}{\mathfrak{g}}
\newcommand{\X}{\mathfrak{X}}
\newcommand{\hf}{\mathfrak{h}}
\newcommand{\glf}{\mathfrak{gl}}
\newcommand{\of}{\mathfrak{o}}

\renewcommand{\Im}{\operatorname{Im}}
\renewcommand{\O}{\operatorname{O}}
\renewcommand{\S}{\mathbb{S}}
\renewcommand{\T}{\mathbb{T}}
\DeclareMathOperator{\Lie}{\operatorname{Lie}}

\DeclareMathOperator{\img}{img}
\DeclareMathOperator{\Arg}{Arg}
\DeclareMathOperator{\End}{End}
\DeclareMathOperator{\I}{I}
\DeclareMathOperator{\id}{id}
\DeclareMathOperator{\Id}{Id}
\DeclareMathOperator{\Alt}{Alt}
\DeclareMathOperator{\sgn}{sgn}
\DeclareMathOperator{\supp}{supp}
\DeclareMathOperator{\Int}{Int}
\DeclareMathOperator{\Ob}{Ob}
\DeclareMathOperator{\Mor}{Mor}
\DeclareMathOperator{\Top}{Top}
\DeclareMathOperator{\CGWH}{CGWH}
\DeclareMathOperator{\Hom}{Hom}
\DeclareMathOperator{\Map}{Map}
\DeclareMathOperator{\Tot}{Tot}
\DeclareMathOperator{\Vect}{Vect}
\DeclareMathOperator{\VectBund}{VectBund}
\DeclareMathOperator{\Open}{Open}
\DeclareMathOperator{\Ring}{Ring}
\DeclareMathOperator{\Set}{Set}
\DeclareMathOperator{\GL}{GL}
\DeclareMathOperator{\SL}{SL}
\DeclareMathOperator{\SO}{SO}
\DeclareMathOperator{\U}{U}
\DeclareMathOperator{\SU}{SU}
\DeclareMathOperator{\Sp}{Sp}
\DeclareMathOperator{\M}{M}
\DeclareMathOperator{\Aut}{Aut}
\DeclareMathOperator{\PGL}{PGL}
\DeclareMathOperator{\PSL}{PSL}
\DeclareMathOperator{\St}{St}

\begin{document}
\begin{minipage}{\textwidth}
	\begin{minipage}{.5\textwidth}
		Complex Manifolds in Dimension 1\\
	\end{minipage}%
	\begin{minipage}{.5\textwidth}
		\raggedleft
		Daniel González Casanova Azuela\par
		{\small\href{https://github.com/danimalabares/riemann-surfaces}{github.com/danimalabares/riemann-surfaces}}
	\end{minipage}%
\end{minipage}\vspace{.2cm}\hrule
\section{Home Assignment 3: Lie groups}
\setcounter{section}{3}
\begin{defn}
	A \textit{\textbf{Lie group}} is a smooth manifold equipped with a group structure such that the group operations are smooth. Lie group $G$ \textbf{\textit{acts on a manifold}} $M$ if the group action is given by the smooth map $G \times M \to M$.
\end{defn}
\begin{exercise}
	Prove that $\SL(n,\R)$ is a Lie group. Prove that it is connected.
\end{exercise}
\begin{proof}\leavevmode
	
	\textbf{($\SL(n,\R)$ is a Lie group.)} (Idea from \cite{lee}) Since $\SL(n,\R)$ is the subgroup of $\GL(n,\R)$ of matrices with determinant $1$, it is the preimage of $\{1\}$ under the smooth function $\det:\GL(n,\R)\to\R$. In fact, $1$ is a regular value of $\det$ because $\det$ is surjective and of constant rank $\equiv1$, making $\SL(n,\R)$ a submanifold. (Of course, $\GL(n,\R)$ is a submanifold of $\R^{2n}=\M(n,\R)$ because it is an open subset, namely, the preimage of $\R\backslash0$ under the continuous function $\det$.)
	
	Moreover, we may think of $\det$ as a group homomorphism from $\GL(n,\R)$ to the multiplicative group $\R\backslash0$, so that $\SL(n,\R)=\ker\det$, making it a subgroup. The restriction of the group operations from $\GL(n,\R)$ are smooth, making $\SL(n,\R)$ a Lie group.
	
	\textbf{($\SL(n,\R)$ is connected.)} (Idea from \href{https://math.stackexchange.com/questions/4774964/prove-that-operatornamesln-bbb-r-is-connected-using-the-decomposition}{StackExchange}) We have seen in our lectures that for any vector space $V$ we have $\SL(V)=e^{\End_0(V)}$ where $\End_0(V)$ denotes the space of matrices with trace 0. We may show that $\SL(n,\R)$ is path-connected by taking any matrix $e^{X}\in\SL(n,\R)$ with $X\in\End_0(\R^n)$ and connecting it to the identity element by the path $e^{tX}$.
\end{proof}

\begin{exercise}\label{ex:3.2}
	Prove that the special unitary group $\SU(n)$ acts transitively on the projective space $\C P^{n-1}$. Find the stabilizer $\St_x(\SU(n))$ of a point $x \in \C P^{n-1}$. Prove that it is connected, or find a counterexample.
\end{exercise}
\begin{proof}\leavevmode

	\textbf{($\SU(n)$ acts transitevly on $\C P^{n-1}$.)} Any point in $\C P^{n-1}$ has a whole circle of representants in the unit sphere $S^{2n-1}\subset \C^n$. Indeed, suppose $x=z_1:\ldots:z_n$ is a point of $\C P^{n-1}$. Since not all coordinates are zero, we may normalize dividing by $\sqrt{z^2_1+\ldots+z^2_n}$, or perhaps by $h(x,x)$. Then for every $\lambda\in S^1$, the point $\lambda(z_1,\ldots,z_n)\in\C^n$ is also a representant of $x$ in $S^{2n-1}$.
	
	Anyway, a matrix $U\in\SU(n)$ not only will preserve $S^{2n-1}$, but will act transitively on it. This follows from Gram-Schmidt orthogonalization process and from \href{https://en.wikipedia.org/wiki/Hadamard%27s_inequality}{Hadamard's inequality}. The latter says that the determinant of a matrix equals the product of the column vectors if they are orthogonal.
	
	Explicitly, we proceed as follows. Any point on the sphere may be extended to an othonormal basis, which is equivalent to a matrix in $\SU(n)$ by Hadamard's inequality. Given any two points on the sphere, the composition of their correponding matrices (using the inverse of one of them) takes one point to another. Transitivity on $S^{2n-1}$ implies transitivity on $\C P^n$.
	
	\textbf{(Find $\St_x(\SU(n))$.)} Consider any representant $z\in S^{2n-1}$ of $x\in\C P^{n-1}$. Elements in $\SU(n)$ that fix $z$ will of course fix $x$. And for each of them, multiplying by $\lambda\in S^{1}$ will yield another operator that fixes $z$ as well.
	
	Now the elements in $\SU(n)$ that fix $z$ can be identified with $\SU(n-1)$. This can be easily seen for the particular case of a simple vector such as $v=(1,0,\ldots,0)$ noticing that the unitary matrices that fix it must have $v$ as a vector in their first column and row, leaving the remaining $n-1$ entries corresponding to a matrix in $\SU(n-1)$. Up to a change of coordinates, this is true for any point in $S^{2n-1}$.
	%\href{https://math.stackexchange.com/questions/3592501/show-that-the-orthogonal-group-acts-transitively-on-the-sphere-sn}{this argument}
	
	We conclude that $\St_x(\SU(n))=\SU(n-1)\otimes S^1$.
	
	\textbf{($\SU(n)$ is connected.)} I have found two proofs:
	
	\textbf{(Proof from \cite{hall}, prop. 1.13)} Any matrix in $U\in\SU(n)$ can be diagonalized with eigenvalues of norm 1. This just means that $U=U_1DU_1^{-1}$ for $U_1\in\SU(n)$ and $D$ diagonal. Now consider
	\[D(t)=\begin{pmatrix}
		e^{\sqrt{-1}(1-t)\theta_1}&&0\\
		&\ddots&\\
		0&&e^{\sqrt{-1}(1-t)\theta_n}
	\end{pmatrix}\]
	Then the path $U(t)=U_1D(t)U_1^{-1}$ connects $U$ and $\Id$ as $t$ moves from 1 to 0. To make sure the path stays within $\SL(n)$ we redefine the last entry to be the inverse of the product of the first $n-1$ entries. This proves that $\SU(n)$ is connected.
	
 	\textbf{(Proof using homogeneous spaces.)} We have seen in our lectures that a transitive action of a Lie group $G$ on a manifold $M$ yields $M=G/St_x(G)$. Then we have that $\C P^{n-1}=\SU(n)/(\SU(n-1)\otimes S^1)$.
 	
 	Another way to put this is in the form of a fiber bundle
 	\[\begin{tikzcd}
 		\SU(n-1)\otimes S^1\arrow[r]&\SU(n)\arrow[r]&\C P^{n-1}
 	\end{tikzcd}\]
 	which in turn yields
 	\[\begin{tikzcd}[row sep=tiny]
 		\pi_0(\SU(n-1)\otimes S^1)\arrow[r]&\pi_0(\SU(n))\arrow[r]&0.
 	\end{tikzcd}\]
 	\iffalse and
	\[\begin{tikzcd}[row sep=tiny]
 		\pi_1(\SU(n-1)\otimes S^1)\arrow[r]&\pi_1(\SU(n))\arrow[r]&\pi_1(\C P^{n-1})=0.
 	\end{tikzcd}\]\fi
 	The final ingredient was found in \cite{piccione}, example 3.2.25: do induction on $n$. Since $\SU(1)=0$, then $\SU(n-1)\otimes S^1$ is connected for all $n>1$ and so is $\SU(n)$.
\end{proof}

\begin{defn}
	Let $W$ be an $n$-dimensional complex vector space equipped with a complex-linear non-degenerate quadratic form $s$. Consider the \textbf{\textit{complex orthogonal group}} $\O(n,\C)$ of all matrices $A \in \GL(W )$ preserving $s$. A subspace $V \subset W$ is called \textbf{\textit{isotropic}} if $s|_V = 0$. It is called \textbf{\textit{maximally isotropic}}, or \textbf{\textit{Lagrangian}}, if $\dim V = [n/2]$.
\end{defn}

\begin{exercise}
	Prove that $\SO(n, \C) := \O(n, \C) \cap \SL(n, \C)$ is a Lie group
	which has index 2 in $\O(n, \C)$. Prove that it is connected.
\end{exercise}
\begin{proof}
	We know  that $\SO(n,\C)$ is a Lie group from our lectures. Now the function $\det:\O(n,\C)\to\{\pm1\}=\Z/2$ is a surjective group homomorphism with kernel $\SO(n,\C)$, meaning $[\O(n,\C):\SO(n,\C)]=|\Z/2|=2$.
	
	This yields a fibration
	\[\begin{tikzcd}
		\SO(n,\C)\arrow[r]&\O(n,\C)\arrow[r]&\{\pm1\}
	\end{tikzcd}\]
	yielding
	\[\begin{tikzcd}
		\pi_0(\SO(n,\C))\arrow[r]&\pi_0(O(n,\C))\arrow[r]&\pi_0(\{\pm1\})
	\end{tikzcd}\]
	and since both $\O(n,\C)$ and $\{\pm1\}$ have two connected components, we have the arrow on the right is an isomorphism and thus $\SO(n,\C)=0$ by exactness of the sequence.
	
	\textbf{(Alternative approach following \cite{hall})} The complex-linear quadratic form $s$ is associated to a symmetric bilinear form, which means we can diagonalize any matrix in $\SO(n,\C)$ with eigenvalues of norm 1 and construct a path to the identity matrix just as in the previous exercise.
\end{proof}

\begin{exercise}
	Let $X$ be the space of all maximally isotropic subspaces in $W$ (the \textbf{\textit{maximally isotropic Grassmanian}}).
	\begin{enumerate}[label*=(\alph*)]
		\item Prove that $\O(n,\C)$ acts on $X$ transitively.
		\item Prove that $X$ is disconnected for $n=2,3$.
		\item Prove that it is connected for $n\geq4$, or find a counterexample.
	\end{enumerate}
\end{exercise}

\begin{proof}\leavevmode
	\begin{enumerate}[label*=(\alph*)]
		\item The idea here is to construc a Hermitian metric as a sum of the symmetric and a skew-symmetric form as follows:
		\[h=g-\sqrt{-1}\omega.\]
		This should be possible given the natural complex structure on $V$ and the bilinear form associated to $s$.
		
		{\color{red}I am unsure wether $s$ is associated to a symmetric or a skew-symmetric bilinear form: while I initially though it would be just a symmetric form, it appears that Grassmanian Lagrangian subspaces are usually defined as those where the skew-symmetric form $\omega$ vanishes. 
		
		In what follows I suppose $s$ is associated to a symmetric bilinear form $g$.}
		
		Now consider two Lagragian Grassmanians $L_1,L_2\in X$ and fix orthonormal bases $(b_j)_{j=1}^n$ and $(b'_j)_{j=1}^n$ with respect to $\omega$. Since $s$ vanishes in both spaces, the restriction of $h$ to either of them becomes $\sqrt{-1}\omega$ and thus these bases are $h$-orthonormal. Then we use transitivity of hermitian forms on bases.
		\item[(b, c)] \iffalse An interesting example (\cite{hatcher}, 4.53) of a fiber bundle is $S^1\to S^{2n+1}\to\C P^n$, given by the map that sends a point to its equivalence class and, as noted in \cref{ex:3.2}, has fiber $S^1$. Surprisingly, this is the case $n=1$ for the more general fiber bundle $\U(n)\to V_n(\C^k)\to G_n(\C^k)$ where $V_n(\C^k)$ is the space of $n$-tuples of orthonormal vectors in $\C^k$ and $G_n(\C^k)$ is the space of $n$-subspaces in $\C^k$. The arrow in the right assigns to a $n$-tuple the vector space it generates, so the fiber is given by all the orthonormal $n$-tuples that generate any given linear space, which is precisely the group of $\U(n)$ by the Gram-Schmidt process.\fi
		
		If the previous exercise is correct, the stabilizer $\St_A$ of every maximally isotropic subspace $V\in X$ is the group of matrices that preserve the skew-symmetric form $\omega$, that is, the symplectic group $\Sp([n/2],\C)$. We have a fiber bundle
		\[\begin{tikzcd}
			\Sp([n/2],\C)\arrow[r]&\O(n,\C)\arrow[r]&X(n)
		\end{tikzcd}\]
		where $X(n)$ is just $X$ for dimension $n$. This gives
			\[\begin{tikzcd}[row sep=tiny]
			\pi_0(\Sp([n/2],\C)\arrow[r]&\pi_0(\O(n,\C))\arrow[r]&\pi_0(X(n))\arrow[r]&0.
		\end{tikzcd}\]
		Since the symplectic group is connected for all $n$ (\cite{piccione} example 3.2.25), we get $\pi_0(X(n))=\pi_0(\O(n,\C))$ for all $n$, meaning $X(n)$ has two connected components for all $n$.
		
		{\color{red} If we associate the quadratic form $s$ to a skew-symmetric form, we get the analogous construction for $\U(n)$ instead of $\Sp([n/2],\C)$ and the result still holds.}


	\end{enumerate}
\end{proof}

\printbibliography
\end{document}